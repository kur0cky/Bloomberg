\documentclass[11pt]{jreport}
\usepackage[dvipdfmx, hiresbb]{graphicx}
\usepackage[top=30truemm, bottom=30truemm, left=30truemm, right=30truemm]{geometry}
\usepackage{float}
\usepackage{ascmac}
\usepackage{amsmath}
\usepackage{amssymb}
\usepackage{bm}
\usepackage{amsfonts}
\begin{document}
\title{Bloomberg Global Investment Contest\\}
\author{東京理科大学工学部第一部経営工学科\\黒木 裕鷹}
\maketitle
\begin{abstract}

\end{abstract}


\chapter{ルールに基づいた戦略の決定}
\section{ルールの概観}

本コンテストで行うシミュレーションは一般的に行われている金融取引とは異なる.最終的な目標は一定のルールのもとでより高い収益を獲得することであるため,投資の戦略もルールに基づいたものでなくてはならない.ルールを概観し,以下にまとめた.

\begin{enumerate}
\item ルール
\begin{enumerate}
\item ポートフォリオ登録時点で時価総額が1億円以内になるように登録
\item 10銘柄以上、30銘柄以下でのポートフォリオを構成
\item 2017年7月31日までであれば1度だけ銘柄の入れ替えが可能
\item ロングポジションのみ(空売り禁止)
\item 手数料は考慮しない
\end{enumerate}

\item パフォーマンス計測期間
\begin{enumerate}
\item  2017年7月3日 $\sim$ 2017年8月31日
\end{enumerate}

\item パフォーマンス測定方法
\begin{enumerate}
\item ポートフォリオ機能「トータルリターン(\%)」を頻度日次・円建てで計測
\item ポートフォリオ登録は登録日より2営業日前の終値をコスト価格として登録
\item ポートフォリオ登録後、6月中の価格変動による時価総額の増減を含め7月3日から8月31日の間でパフォーマンスを計測
\end{enumerate}
\end{enumerate}

以上のルールの中で,時間的に制約を受けるものは1-c, 2-aである.これらにより1か月単位のバイアンドホールドを強いられ,さらにルール3-bにより裁定機会が失われることになる.つまり,金融資産の価格変動は何を根拠として起こっているのか,という視点で考える必要があり,それに基づいて価格が上昇する銘柄を追い求めていくことになる.次の1.2節では,価格変動の源泉について考察した.

\section{価格変動の源泉}
金融資産は一般的な商品と異なり、投資目的やリスクヘッジ目的で購入されることがほとんどであるから,その価格は需要供給の関係だけでなく将来の見通しにも影響される。つまり、「情報」が価格に織り込まれるのである。すると、未だ織り込まれていない「情報」を見つけ出して特定の銘柄を購入し、織り込まれた時点で売却する、という戦略が考えられる。この手法は一見有効であるように思えるが、ここには一つ盲点ある。それは、証券アナリストや裁定取引屋をはじめとする高度に習熟したプロフェッショナルの存在である。習熟したプロ達はいち早く織り込まれていない情報を日々見つけ出し、ポートフォリオに組み込むことを目標にしている。アナリスト達が織り込んでいない情報を見つけ出す手段として、アナリスト達が注目していない、またはポートフォリオに組み込めないような市場や超小型銘柄に焦点を当てる手法が考えられる。しかしルール1-c, 2-a, 3-bの存在により、この手法も困難であることが分かる。織り込まれていない情報が織り込まれたとして、8月31日までその価格が継続する可能性は非常に低い。また、測定期間内にその情報が織り込まれる可能性も低いといえる。

次の1.3節では別の見方から価格変動を眺める.
 
\section{もう一つの源泉}
世界には無数の企業があり,それぞれが異なった性質を持っている.持っている技術やリリースしている商品,組織体系などが多種多様であるということだ.しかし,複数の企業に共通している性質も考えられる.例えば所属国や業種,企業の規模などだ.このような共通の要因のうち,株価の変動に関係を持つものも当然存在するだろう.FamaとFrenchは米市場における実証分析を行い,企業規模(size)や簿価比時価率(Value)が株価の収益構造に関与していると結論付けた.つまり,企業規模の小さい銘柄や,時価総額が総資産額に比べて割安な銘柄は平均的に高い収益を見せるということだ.この現象はそれぞれ小型株効果,バリュー株効果と呼ばれ,代表的なアノマリーとして認識されている.先ほどのモデルはFama-Frenchの3ファクターモデルと呼ばれ,FamaとFrenchは2013年にその功績を称えられノーベル経済学賞を受賞した.実際に似たような性質(業種や企業規模など)の銘柄は似たような価格変動を見せることが多く,FamaとFrenchが対象とした米市場に限らず,金融資産の価格変動が共通の要因に影響されている,という主張は自然なものだと考えられる.

\section{様々な理論}
ポートフォリオ選択の理論としてはマーコヴィッツの平均分散モデルが,資産の適正価格を決定する理論としてはシャープらの資本資産価格モデル(capital asset pricing model : CAPM)が1960年代より不動の地位を築いてきた.計算の簡便さもあり,これらの理論は現在でも広く用いられている.しかし1970年代以降,CAPMに対する様々な批判や問題点が提起され,代わりとなる新たな理論が提唱され続けている.CAPMが必要とする仮定(2章で詳しく述べる)が非常に限定的であり,到底成立するものではない,ということである.そこで,本コンテストではCAPM以降に広く用いられてきた裁定価格理論(arbitrage pricing theory : APT)を用いることとした.

詳細については後に触れるが,APTはCAPMと異なり,全資産の収益の同時分布が正規分布であることを必要しないのである.また,これも後述するが,APTはファクターモデルと非常に相性が良い.

\section{戦略の決定}
1.2節で述べたように、これから価格に織り込まれる「情報」を追うのではなく違う視点から戦略を決定する必要がある.また1.3節で述べた,Fama-Frenchの3ファクターモデルをはじめとするマルチ・ファクターもでるにより,CAPMよりも確からしい資産の価格変動の構造を考えることができる.そこで独自のマルチファクターモデルを構築し,「直近で平均的に勝てている投資スタイル」を見つけ出すことが出来れば2ヶ月間の収益を競う本コンテストにおいても成果を上げることが出来るのではないかと考えた.なお,本コンテストは2カ月間のみの短いものであり,偶発的な金融危機については考えないこととする.とはいえ,収益に見合わないリスクを持つポートフォリオを構成することには何の利点も考えられないため,特に理由がない限り構成銘柄数は最大の30銘柄を考える.

市場には無数の裁定取引を行う投資家が存在すること,さらにコンテストのルールにより,鞘取りに関しては全く行うことが出来ない.市場に裁定機会が存在しないことを主な仮定とするAPTと非常に相性の良いものであり,本コンテストのルールの下で収益を出すために十分有用であると考えられる.

マルチ・ファクターモデルにより各銘柄の収益構造を推定した後の問題はポートフォリオに組み込む資産をどのように選択するかである.投資には必ずリスクが伴い,投資家はそのリスクを代償にリターンを求める.そこでここでは,それぞれのファクターの持つリスクに晒される「価値」を考えることとした.この「価値」は通常リスク・プレミアムと呼ばれる.高いリスク・プレミアムを持つファクターに対してリスクを取り,低いリスク・プレミアムを持つファクターに対して分散化することを考えていく.


\chapter{理論}
\section{シングル・ファクターモデルとしてのCAPM}
CAPMによれば,金融資産$i$の収益率$r_i$がマーケットの収益率$R_M$に依存する部分と資産特有の部分$\varepsilon_i$に分けられ,以下のように表される.また,$\varepsilon_i$は資産間で相互に無相関であるとされる.%企業特有の部分であってる?
\begin{equation}
\begin{split}
&r_i = \alpha_i + \beta_iR_M + \varepsilon_i\\
&\varepsilon_i \sim i.i.d.N(0,\sigma_i^2)\\
&Cov(R_M, \varepsilon_i) = 0
\label{eq:CAPM}
\end{split}
\end{equation}
ここで,$\beta_i$は金融資産$i$のマーケットへの感度を表し,$\alpha_i$はマーケットに対する期待超過収益率を表している.式(\ref{eq:CAPM})より$r_i$の分散を求めると
\begin{equation}
Var(r_i) = \beta_i^2Var(R_M) + Var(\varepsilon_i) \qquad (\text{∵}Cov(R_M, \varepsilon_i) = 0)
\label{eq:CAPM_var}
\end{equation}
となり,この右辺第1項はシステマティック・リスクと呼ばれ,シングル・ファクターであるマーケットの動きにより説明可能な部分である.また右辺第2項はマーケットに依存しないアンシステマティック・リスクと呼ばれる.
このうちアンシステマティック・リスクはポートフォリオ選択によって除去可能であるとしばしば言われるが,Fama-Frenchの3ファクターモデルに代表される数々の実証分析ではこれが不可能であることが報告されている.このことは,マーケットではない共通要因も存在し,それらがアノマリーの原因となっていることを示唆している.

市場を構成する資産数を$N$とすると,資産$i(i=1,\cdots,N)$の保有比率が$w_i$であるようなポートフォリオ$P$の収益率$R_P$は以下で与えられることになる.
\begin{equation}
\begin{split}
R_P &= \alpha_P + \beta_PR_M + \varepsilon_P\\
&= \sum_{i=1}^N w_i\alpha_i
+\left(\sum_{i=1}^N w_i\beta_i\right)R_M
+\sum_{i=1}^N w_i\varepsilon_i
\end{split}
\end{equation}
また,このポートフォリオの分散は以下で与えられる.
\begin{equation}
\begin{split}
Var(R_P) &= \beta_P^2 Var(R_M) + Var(\varepsilon_P)\\
& = \left(\sum_{i=1}^N w_i\beta_i\right)^2Var(R_M) + \sum_{i=1}^N w_i^2 Var(\varepsilon_i)
\end{split}
\end{equation}
この$\beta_P$がポートフォリオのマーケットに対する感度になり,$\alpha_P$がリスク・プレミアムとなる.
%リスク・プレミアムでいいの?
\section{Fama-Frenchの3ファクターモデル}
%本当に必要??あとまわし

\section{マルチ・ファクターモデル}
ここでは,各資産の収益率が$m$個の共通要因に依存するマルチ・ファクターモデルを考える.金融資産$i$の収益率$r_i$は企業特有の部分$\varepsilon_i$と共通要因であるファクター$F_k(k=1,\cdots,m)$によって決定される.また,$\varepsilon_i$は資産間で相互に無相関であると仮定すると,$r_i$は以下のように表される.
\begin{equation}
r_i = \alpha_i + \beta_{i,1}F_1 + \beta_{i,2}F_2 + \cdots + \beta_{i,k}F_k + \cdots + \beta_{i,N}F_m + \varepsilon_i
\end{equation}
ここで,2.1節と同様,資産$i(i=1,\cdots,N)$の保有比率が$w_i$であるようなポートフォリオ$P$の収益率$R_P$を考えると以下のようになる.
\begin{equation}
\begin{split}
R_P &= \alpha_P + \beta_{P,1} F_1 + \beta_{P,2} F_2 + \cdots + \beta_{P,m} F_m + \varepsilon_P\\
&=\sum_{i=1}^N \alpha_i + F_1 \sum_{i=1}^N w_i \beta_{i,1} + \cdots + F_m \sum_{i=1}^N w_i \beta_{i,m} + \sum_{i=1}^N w_i\varepsilon_i\\
&= \sum_{i=1}^N \alpha_i + \sum_{k=1}^m \sum_{i=1}^N w_i \beta_{i,k} F_k + \sum_{i=1}^N w_i\varepsilon_i
\label{eq:multi}
\end{split}
\end{equation}

%次に,式(\ref{eq:multi})の分散を考える.
%ポートフォリオの分散は求めたほうが良いの?
\section{APT}
%ルーエンバーガーの本
%ファクターのプレミアムとかみ合わない話がある
%どーする
\section{ファクターのリスク・プレミアムの算出}
資産$i$の収益率が式(\ref{eq:multi})で表されているとする.このとき資産を組み合わせて,特定のファクターへの感度のみが1で,その他のファクターへの感度が0になるポートフォリオを考える .例えば,第1ファクターのみへの感度が1になるようなポートフォリオを$P_1$とすると,その収益率は
\begin{equation}
R_{P_1} = \alpha_{P_1} + 1\times F_1 + 0\times F_2 + \cdots + 0\times F_m + \varepsilon_{P_1} 
\end{equation}
と表せる.このとき,$\alpha_{P_1}$が第1ファクターのリスク・プレミアムとなる.

次に,各ファクターのリスク・プレミアムを行列計算により求めることを考える.なお2.3節と同様,N資産に対するmファクターモデルを考える.第$k(k=1,2,\cdots,m)$ファクターに対する感度のみが1であるポートフォリオを第kファクター・ポートフォリオ$P_k$とし,$P_k$における資産$i$のウェイトを$w_{i,k}$とする.さらに,資産$i$の第$k$ファクターへの感度を$\beta_{i,k}$とすると,以下の式(\ref{eq:fp})のようにまとめることが出来る.

\begin{equation}
\left(
	\begin{array}{cccc}
	w_{1,1} & w_{1,2} & \ldots & w_{1,m}\\
	w_{2,1} & w_{2,2} & \ldots & w_{2,m}\\
	\vdots & \vdots & \ddots & \vdots\\
	w_{n,1} & w_{n,2} & \ldots & w_{n,m}
	\end{array}
\right)^{\mathrm{T}}
\left(
	\begin{array}{cccc}
	\beta_{1,1} & \beta_{1,2} & \ldots & \beta_{1,m}\\
	\beta_{2,1} & \beta_{2,2} & \ldots & \beta_{2,m}\\
	\vdots & \vdots & \ddots & \vdots\\
	\beta_{n,1} & \beta_{n,2} & \ldots & \beta_{n,m}\\
	\end{array}
\right)=
\left(
	\begin{array}{cccc}
	1 & 0 & \ldots & 0\\
	0 & 1 & \ldots & 0\\
	\vdots & \vdots & \ddots & \vdots \\
	0 & 0 & \ldots & 1
	\end{array}
\right)
\label{eq:fp}
\end{equation}


\chapter{実際のデータ分析}
\section{使用したデータ}
以下のデータを2015年1月$\sim$2017年5月の期間で,日次終値の形で取得した.なお,データの取得にはBloomberg端末とMicrosoft ExcelのBloombergアドインを使用し,データ分析はR言語により行った.
\begin{itemize}
\item 資産価格について\\
東証一部上場企業の株式の価格データを使用した.
\item ファクターについて\\
マルチ・ファクターモデルを構成する際,以下のファクターを使用した.
\begin{itemize}
\item マーケットファクター\\
東証一部上場銘柄を対象としているため,マーケットファクターとしては東証一部上場銘柄の時価総額を反映しているTOPIXを使用した.変動の比率を見るため,対数収益率を取った.
\item サイズファクター\\
Russell/Nomuraの提供する日本株インデックスで代用した.具体的には,配当を含めないSmall Cap Indexから配当を含めないLarge Cap Indexを引くことにより算出した.さらにその変動をみるために差分を取ったものを使用した.
\item バリューファクター\\
Russell/Nomuraの提供する日本株インデックスで代用した.具体的には,配当を含めないTotal Value Indexから配当を含めないTotal Growth Indexを引くことにより算出した.さらにその変動をみるために差分を取ったものを使用した.
\item 為替ファクター\\
日本円(JPY)の市場価格(対米ドル)を為替ファクターとして使用した.さらにその変動をみるために差分を取ったものを使用した.
\item ボラティリティファクター\\
S\&PがVIXと同様の手法で算出している日本市場の30日インプライド・ボラティリティを使用した.
\end{itemize}
上記のファクターはそれぞれ単位もスケールも異なるので,正規化(平均0分散1に標準化)した後に使用した.
\end{itemize}
\section{データのクリーニング・概観}
対象期間とした2015年1月$\sim$2017年5月において,何らかの理由(そもそもまだ出来ていない,上場していない,等の)により価格データが欠けている企業が見受けられたため,当該企業を除外した.さらに対数収益率を取ることにより全営業日との比率を算出した.

対象銘柄すべての価格推移をプロットすることは不可能であるため,使用するファクターについてのみ時系列プロットを行い,以下の図に示した.また,多重共線性などの問題が存在するならばそれを回避するため,ファクターの散布図を以下の図に,相関行列を算出し以下の表にまとめた.
%対数収益率について
\section{ファクターモデルの推定}
扱いやすい日次終値データを対象としていることに加え時系列データの性質により,推定するパラメータ数に対し十分な標本が得られないという問題がある.また市場の性質が不変であるとは考えにくく,ゆるやか(もしくは何らかのショックにより急激に)変化していくものと思われる.そのためバックテストも見据え,3カ月分のデータを用いてモデルを推定し,その先1カ月を対象にバックテストを行うこととした.

マルチ・ファクターモデルを推定する際,Fama-Frenchの3ファクターモデルやCarhartの4ファクターモデルのような線形モデルだけでなく,非線形モデルも考慮に入れることができる.しかしながら,標本の少なさや時系列データにおける外挿の問題により複雑なモデルを考えるメリット小さく,ここでは線形モデルのみを考えた.

線形でのマルチ・ファクターモデルを推定する際にも注意しなければならないことがある.むやみに説明変数(ここでは各ファクターを指す)を増やすことにより生じる多重共線性の問題や多重性の問題,みせかけの回帰の問題である.そこで,説明変数を全て使用しての線形回帰ではなく,何らかの工夫を施すことによりむやみにモデルを複雑にしない操作が必要になる.選択肢としては以下の手法が考えられる.

\begin{itemize}
\item AIC(マローの$C_p$と同義)やBICなどの情報量基準を用いたモデル選択
\item リッジ回帰やlasso回帰などの,正則化項を用いたモデル推定
\item 介入効果と回帰係数を一致させるための統計的因果推論に基づいた説明変数選択
\end{itemize}

\subsection{lasso回帰について}
\section{リスク・プレミアム}
一般化逆行列
\section{バックテスト}
様々なポートフォリオ
\chapter{途中経過}
\section{提出したポートフォリオ}

\begin{table}
\begin{center}
\begin{tabular}{|c|c|c|}
\hline
企業名 & ウェート(\%) & 時価総額(円, 7月3日時点)\\
\hline
\hline
グローブライド	&3.34&3,343,642\\
コメダホールディングス&3.33	&3,335,200\\
ゼンショーホールディングス&3.33&	3,339,875\\
三井ホーム&3.33&3,333,333\\
井筒屋&3.33	&3,341,014\\
日成ビルド工業&3.35&3,360,302\\
美津濃&3.28	&3,287,108\\
藤倉ゴム工業	&	3.36	&	3,369,011	\\
近鉄百貨店	&	3.35	&	3,352,435	\\
GSIクレオス	&	3.30	&	3,309,523	\\
イマジカ・ロボット・ホールディングス	&	3.34	&3,351,877\\
グランディハウス	&	3.30	&	3,309,859	\\
サンフロンティア不動産	&	3.43	&	3,432,881	\\
北陸電力	&	3.32	&	3,326,772	\\
アイ・オー・データ機器	&	3.35	&	3,360,768	\\
アドソル日進	&	3.36	&	3,363,838	\\
システムリサーチ	&	3.36	&	3,363,148	\\
ルネサスイーストン	&	3.36	&	3,367,816	\\
日本電波工業	&	3.36	&	3,363,984	\\
ソフトクリエイトホールディングス	&	3.35	&3,361,324\\
カワチ薬品	&	3.35	&	3,353,240	\\
三井製糖	&	3.39	&	3,395,734	\\
六甲バター	&	3.30	&	3,305,533	\\
森永乳業	&	3.23	&	3,234,511	\\
養命酒製造	&	3.31	&	3,317,285	\\
ニチハ	&	3.34	&	3,350,168	\\
三井住友建設	&	3.30	&	3,306,233	\\
太平洋興発	&	3.33	&	3,333,333	\\
極東証券	&	3.28	&	3,285,697	\\
SOMPOホールディングス	&	3.35	&	3,353,441\\
\hline
\hline
合計& 100.00 &  100,208,884\\
\hline
\end{tabular}
\end{center}
\caption{提出したポートフォリオ}
\label{tbl:port1}
\end{table}
\section{7月中のパフォーマンス}
\chapter{課題とリバランス}
\end{document}